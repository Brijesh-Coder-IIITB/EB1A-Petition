% Evidence of Membership in Associations Requiring Outstanding Achievements
% 8 CFR § 204.5(h)(3)(v)

% Define blue color command for recommendation letter quotes if not already defined
\providecommand{\rlquote}[1]{\textcolor{blue}{#1}}

\textbf{Pursuant to 8 CFR § 204.5(h)(3)(v)}, \dr has established membership in associations in his field that require outstanding achievements of their members, as judged by recognized national and international experts in the discipline. \dr holds \ul{three prestigious memberships}: (1) \textbf{Fellow of BCS, The Chartered Institute for IT (FBCS)}, the United Kingdom's premier professional body for IT professionals operating under Royal Charter; (2) \textbf{IEEE Senior Member}, the world's largest technical professional organization; and (3) \textbf{ISSIP Ambassador}, an appointment conferred through nomination and Executive Leadership approval by the International Society of Service Innovation Professionals (ISSIP) .

%% ===========================================================================
%% EXECUTIVE SUMMARY
%% ===========================================================================

\vspace{0.5em}
\textbf{Key Evidence of Outstanding Achievement:}
\begin{itemize}
    \item \textbf{BCS Fellow (FBCS):} Granted 2024---represents \ul{less than 5\% of BCS's 65,000 members} across 150 countries , 
    \item \textbf{IEEE Senior Member:} Elevated October 11, 2023---reserved for \ul{only 8--10\% of IEEE's 460,000+ members} 
    \item \textbf{ISSIP Ambassador:} Appointed July 4, 2024---selected via nomination and Executive Leadership review/approval; ISSIP reports \ul{120 current Ambassadors} , 
    \item \dr holds \ul{62 granted United States patents} demonstrating sustained technical innovation
    \item Following BCS Fellowship recognition, \dr was selected as \textbf{Mentor at CSULB Institute for Innovation \& Entrepreneurship} 
    \item Team mentored by \dr (CuraVoice) won \textbf{Sunstone Innovation Challenge} (\$15,000 prize) and \textbf{PyTorch 2025 Community Choice Award} , 
\end{itemize}

%% ===========================================================================
%% BCS FELLOWSHIP
%% ===========================================================================

\subsubsection{BCS Fellow: Association Requiring Outstanding Achievements}
\label{member:bcs}

\paragraph{Association Overview and Prestige}

BCS, The Chartered Institute for IT, is the United Kingdom's premier professional body for information technology professionals . BCS operates under a \ul{Royal Charter granted in 1984} by Her Majesty the Queen, which empowers it to confer chartered status upon IT professionals and establish standards of competence, conduct, and ethical practice for the profession. Founded in 1957, BCS has over 60 years of history advancing computing and IT for the benefit of society.

\vspace{0.5em}
Per the BCS 2024 Annual Report , BCS comprises \textbf{approximately 65,000 members across 150 countries}, with 45 regional branches in the UK and 16 international sections. The organization maintains a network of over 50 specialist groups covering diverse IT domains from artificial intelligence to cybersecurity. BCS is \ul{universally recognized} by the UK government, academic institutions, and global industry leaders as the authoritative voice of the IT profession. The Royal Charter status places BCS alongside other distinguished chartered bodies such as the Royal Society and the Institution of Civil Engineers.

\vspace{0.5em}
\textbf{Relevance to \drs Field:} BCS directly addresses \drs field of extraordinary ability---artificial intelligence, machine learning, and computer science. BCS publishes journals, organizes conferences, and maintains specialist groups in AI and data science. \drs work in generative AI systems, enterprise computing, and patent innovation aligns precisely with BCS's mission to advance IT for societal benefit.

\paragraph{Membership Level and Selectivity}

\textbf{Membership Tier:} Fellow of BCS (FBCS)

\vspace{0.5em}
BCS offers the following membership grades (ascending order of prestige):
\begin{enumerate}
    \item \textbf{Associate Member:} Entry-level membership for IT professionals
    \item \textbf{Member (MBCS):} For professionals with demonstrated competence
    \item \textbf{Chartered IT Professional (CITP):} Requires demonstrated professionalism
    \item \textbf{Fellow (FBCS):} \ul{Highest professional grade; requires outstanding achievements; less than 5\% of members}
    \item \textbf{Distinguished Fellow:} Honorary recognition for exceptional lifetime contributions (only 24 total since 1971)
    \item \textbf{Honorary Fellow:} Bestowed upon individuals of exceptional distinction (only 104 total to date)
\end{enumerate}

\ul{\textbf{Fellow (FBCS) is the highest professional grade for which BCS members can apply based on demonstrated merit.}} Distinguished Fellow and Honorary Fellow status are conferred only for exceptional lifetime contributions and are not open to application. Thus, FBCS represents the pinnacle of professional recognition that a member can actively seek and achieve through demonstrated outstanding achievements.

\paragraph{Statistical Evidence \& Selectivity (Source: BCS 2024--2025 Annual Report)}

As of the 2024--2025 reporting period, per the BCS Annual Report 2024 (Page 6) , the statistical evidence demonstrates the highly selective nature of BCS Fellowship:

\vspace{0.5em}
\textbf{BCS Worldwide Membership Distribution:}

\vspace{0.5em}
\begin{tabular}{|p{5.5cm}|p{3.5cm}|p{3.5cm}|}
\hline
\textbf{Membership Category} & \textbf{Approximate Count} & \textbf{Percentage of Total} \\
\hline
Total Worldwide Membership & $\sim$65,000 & 100\% \\
\hline
BCS Fellows (FBCS) & $\sim$2,500 -- 3,000 & $\sim$3.8\% -- 4.6\% \\
\hline
Distinguished Fellows & 24 (total since 1971) & $<$ 0.04\% \\
\hline
Honorary Fellows & 104 (total to date) & $<$ 0.16\% \\
\hline
\end{tabular}

\vspace{0.5em}
The Fellowship (FBCS) grade is the highest professional grade and is \ul{significantly more selective} than the standard ``Member'' (MBCS) grade. While BCS does not publish a real-time count of every grade, historical data and annual reports consistently show that \textbf{Fellows represent approximately 3\% to 5\% of the total membership}---placing them among an exclusive cohort of IT professionals recognized for outstanding contributions to the field.

\vspace{0.5em}
\fbox{\parbox{\textwidth}{%
\textbf{Summary:} \dr is a Fellow of the British Computer Society (FBCS). This is the \ul{highest professional grade} of the UK's Chartered Institute for IT. Per the society's 2024 Annual Report , out of a global membership of approximately 65,000, \textbf{only a small fraction (less than 5\%)} attain the rank of Fellow. Admission to this grade requires a \ul{rigorous peer-review process} and evidence of ``outstanding achievement'' and ``eminence'' in the field, satisfying the regulatory criteria for membership in associations that require outstanding achievements of their members.
}}

\paragraph{Outstanding Achievement Requirements}

BCS Fellowship is \ul{not} conferred based on payment of dues, years of experience alone, or educational credentials. Rather, it requires demonstrated \textbf{outstanding achievements} evaluated against specific criteria established in BCS bye-laws . Candidates must demonstrate excellence across three core evaluation areas:

\vspace{0.5em}
\textbf{BCS Fellowship Evaluation Criteria:}
\begin{enumerate}
    \item \textbf{Body of Work:} Evidence of significant professional achievements in IT, including invention and innovation, entrepreneurship, technical responsibility, research impact, skills development, or consultancy leadership
    
    \item \textbf{Professional Impact:} Demonstrated individual contributions and measurable outcomes, including inspiring others, interdisciplinary collaboration, policy development, diversity and inclusion leadership, mentoring, sustainability impact, or community outreach
    
    \item \textbf{Standing in the Community:} Recognition of contributions by the broader IT community, including awards, governance roles, public influence, or service as an assessor for professional bodies
\end{enumerate}

Applicants must satisfy \ul{at least four sub-criteria across these three areas}, with a minimum of one from each category. This ensures that Fellows have demonstrated outstanding achievements across multiple dimensions of professional excellence.

\vspace{0.5em}
\textbf{How \dr Meets These Requirements (From BCS Fellowship Application):}
\begin{enumerate}
    \item \textbf{Body of Work---Invention and Innovation:} \dr submitted evidence of \ul{65+ patent applications with 62 granted in the United States}. His issued patents have been \textbf{forward-cited} by major technology leaders---including \textbf{Amazon (AWS)}, \textbf{Microsoft}, \textbf{Google}, \textbf{IBM}, \textbf{Dell Technologies}, \textbf{EMC}, \textbf{Hewlett Packard Enterprise (HPE)}, \textbf{Oracle}, \textbf{Seagate}, \textbf{Western Digital}, \textbf{NetApp}, \textbf{Veritas}, \textbf{Commvault}, \textbf{Hitachi}, \textbf{Samsung}, \textbf{Intel}, and \textbf{Micron}---providing objective, third-party evidence that his inventions constitute influential prior art and are \ul{being built upon by the industry’s most significant technology organizations}, confirming major significance in cloud infrastructure, enterprise storage, and cybersecurity .
    
    \item \textbf{Professional Impact---Mentoring and Coaching:} \dr designed and implemented a mentoring program for patent filing across multiple business units. He mentored both senior and same-level colleagues, achieving a \ul{100\% success rate} in patent submissions for all mentees. One mentee was promoted from Senior Principal Engineer to \textbf{Distinguished Member of Technical Staff} with patents as a key criterion . \dr also published a book: \textit{``Conformal Prediction: An Inventor's Approach''} in 2024 based on his extensive patent portfolio and mentorship experience.
    
    \item \textbf{Body of Work---Research:} \dr's research on uncertainty quantification for machine learning (Conformal Prediction) was published at \textbf{SNIA SDC, SYSTOR, KDD, IEEE Access, and DATE}. His hardware security research achieved publication at a top-tier conference with \ul{23\% acceptance rate}. His work received citations from major technology companies including \textbf{Meta Platforms, IBM, Fujitsu, RedHat, Commvault, Kyndryl}, and academic citations from \textbf{Cambridge University (Cavendish Laboratory)}. He delivered \textbf{8 talks at SNIA SDC} (2019--2023) and was ranked \textbf{\#10 inventor in Long Beach, CA} as per iDiyas.com.
    
    \item \textbf{Standing in the Community---Awards:} \dr received the \ul{CSULB Outstanding Graduate Student Award} (selected from 5,500 students) for Research, Scholarly, and Creative Activities. Additional recognitions include: NSF Student Travel Grant, DAC Young Fellows Program (60th Design Automation Conference), Dell Patent Milestone Award (20+ patents), and cumulative \textbf{Dell Patent Awards totaling \$255,610}. He was recognized as \ul{Top Patent Filer across APJ, EMEA, and LATAM} at Dell Technologies in FY21. His IEEE Senior Member status further demonstrates field recognition.
\end{enumerate}

\paragraph{Expert Judgment and Review Process}

BCS employs rigorous peer-review for all Fellowship applications through its Membership Assessment Panel . The review process requires submission of detailed evidence statements demonstrating achievements across the three evaluation areas, endorsement by an existing BCS Fellow or professional body Fellow, and review by assessors who are themselves recognized experts (typically existing Fellows or Chartered professionals). Applications that do not demonstrate outstanding achievements across the required criteria are rejected.

\vspace{0.5em}
\textbf{\drs Application Outcome:} \dr submitted his Fellowship application with comprehensive documentation of his patent portfolio, leadership roles, mentoring activities, and professional recognition. Following review by BCS assessors, \dr was \ul{successfully elevated to BCS Fellow (FBCS)} in 2024 . He received official welcome correspondence from BCS confirming his Fellowship status .

\vspace{0.5em}


\paragraph{Evidence of Genuine Selectivity}

BCS Fellowship is \ul{NOT} a fee-based or credential-based membership:
\begin{enumerate}
    \item \textbf{Application Required:} Comprehensive evidence statements demonstrating outstanding achievements
    \item \textbf{Expert Endorsement Required:} Existing Fellow or equivalent must attest to qualifications
    \item \textbf{Review Panel Evaluation:} Every application evaluated by qualified assessors
    \item \textbf{Rejections Occur:} Applications not meeting the criteria are denied
    \item \textbf{Royal Charter Authority:} BCS operates under Royal Charter, conferring governmental recognition
    \item \textbf{Statistical Selectivity:} Only approximately 3--5\% of 65,000 members achieve Fellow status (less than 3,000 worldwide per 2024 Annual Report) 
\end{enumerate}

\paragraph{Impact of BCS Fellowship: Mentorship and CuraVoice Success}

Following his elevation to BCS Fellow, \dr was selected as a \textbf{Mentor at the California State University, Long Beach (CSULB) Institute for Innovation \& Entrepreneurship (IIE)} . His profile is listed among the mentors for the Sunstone Innovation Challenge, a business planning competition that helps students develop and pitch innovative ideas.

\vspace{0.5em}
Under \drs mentorship, the team \textbf{CuraVoice} won top honors at the 2024 Sunstone Innovation Challenge, a ``Shark Tank''-like exhibition where teams present their business plans to local business leaders . The win came with \$15,000 in prize money and access to professional services. CuraVoice's AI-powered voice simulation technology is designed to emulate vocal characteristics of patients with distinct ailments, helping healthcare students and professionals improve patient communication skills.

\vspace{0.5em}
Subsequently, CuraVoice achieved national recognition at the \textbf{PyTorch 2025 Startup Showcase}, where they won the \textbf{Community Choice Award} . The NVIDIA blog covering Open Source AI Week in San Francisco specifically named \dr as an ``advisor'' to CuraVoice, stating: ``The Community Choice Award was presented to CuraVoice, with CEO [Name], CTO [Name], and \ul{advisor Rahul Vishwakarma} accepting the award on behalf of the team'' . NVIDIA's blog further describes CuraVoice as ``an AI-powered voice simulation platform---powered by NVIDIA Riva for speech recognition and text-to-speech, and NVIDIA NeMo for conversational AI models---for healthcare students and professionals.''

\vspace{0.5em}
This chain of events---BCS Fellowship leading to mentorship selection, leading to successful student outcomes with national recognition---demonstrates that \drs outstanding achievements continue to generate real-world impact and that his expertise is sought for developing the next generation of AI innovators.

\vspace{0.5em}


\paragraph{Independent Verification}

USCIS may verify \drs BCS Fellowship status through:
\begin{enumerate}
    \item \textbf{BCS Fellowship Certificate} 
    \item \textbf{BCS Welcome to Fellowship Correspondence} 
    \item \textbf{BCS Member Directory:} \dr is listed as Fellow (FBCS)
    \item \textbf{CSULB IIE Mentor Page:} \dr is listed at csulb.edu/sunstone-innovation-challenge/mentors 
    \item \textbf{PyTorch Blog:} Documents CuraVoice Community Choice Award 
    \item \textbf{NVIDIA Blog:} Names \dr as advisor to CuraVoice 
    \item \textbf{Contact:} BCS, The Chartered Institute for IT, 3 Newbridge Square, Swindon SN1 1BY, United Kingdom; +44 (0)1793 417 417; customerservice@bcs.uk
\end{enumerate}

%% ===========================================================================
%% IEEE SENIOR MEMBERSHIP
%% ===========================================================================

\subsubsection{IEEE Senior Member: Association Requiring Outstanding Achievements}
\label{member:ieee}

\paragraph{Association Overview and Prestige}

The Institute of Electrical and Electronics Engineers (IEEE) is the world's largest technical professional organization dedicated to advancing technology for the benefit of humanity . Founded in 1963, IEEE has over 140 years of history advancing technological innovation.

\vspace{0.5em}
IEEE comprises \textbf{over 460,000 members across 190+ countries} and publishes approximately \textbf{30\% of the world's technical literature} in electrical engineering and computer science. The organization convenes \textbf{over 1,900 conferences annually} and develops foundational industry standards including IEEE 802 (Wi-Fi, Ethernet) and IEEE 754 (floating-point arithmetic). IEEE is \ul{universally recognized} by academic institutions, government agencies---including the U.S. Department of Energy---and industry leaders as the preeminent organization in electrical engineering and computer science.

\vspace{0.5em}
\textbf{Third-Party Validation of IEEE Senior Member Prestige:}

Oak Ridge National Laboratory (ORNL)---a U.S. Department of Energy premier research facility---regularly publishes press releases when researchers achieve IEEE Senior Member status, including announcements for \textbf{[Name]}, \textbf{[Name]} and \textbf{[Name]}, and \textbf{[Name]} , , . These releases describe IEEE Senior Member as a peer-recognized distinction reserved for only a small fraction of IEEE's global membership:

\begin{quote}
``\textit{This highest level of IEEE membership, awarded by peers for technical and professional excellence, is reserved for only the roughly 10\% of the organization's 450,000 members who have made the most significant contributions to the engineering field over time.}'' 
\end{quote}

\paragraph{Membership Level and Selectivity}

\textbf{Membership Tier:} IEEE Senior Member

\vspace{0.5em}
IEEE offers the following membership grades (ascending order of prestige):
\begin{enumerate}
    \item \textbf{Student/Graduate Student Member:} Open to enrolled students
    \item \textbf{Member:} Basic professional membership
    \item \textbf{Senior Member:} \ul{Requires outstanding achievements; approximately 8--10\% of members}
    \item \textbf{Fellow:} Highest grade; less than 0.1\% of members annually (nomination only)
\end{enumerate}

\ul{\textbf{Senior Member is the highest grade for which IEEE members can apply.}} The Fellow grade is only available through nomination by current Fellows. Thus, Senior Member represents the highest level of recognition that a member can actively seek and achieve based on demonstrated merit.

\paragraph{Outstanding Achievement Requirements}

IEEE Senior Member grade is \ul{not} conferred based on payment of dues, years of experience alone, or educational credentials. Rather, it requires demonstrated \textbf{outstanding achievements} evaluated against specific criteria established in IEEE bylaws :

\vspace{0.5em}
\textbf{IEEE Bylaw Requirements:}
\begin{enumerate}
    \item \textbf{Professional Experience:} Minimum ten years in IEEE-designated fields, with at least five years of significant performance
    \item \textbf{Significant Performance:} Documented contributions that advance the field (not routine work)
    \item \textbf{Reference Endorsements:} Three references from IEEE members, at least two must be Senior Members, Fellows, or Life Members
    \item \textbf{Technical Contributions:} Evidence of publications, patents, leadership roles, or other documented achievements
\end{enumerate}

\vspace{0.5em}
\textbf{How \dr Meets These Requirements:}
\begin{enumerate}
    \item \textbf{Professional Experience:} \dr has accumulated \ul{over 15 years of professional experience} in computer science, artificial intelligence, and machine learning through positions at Dell Technologies, Hewlett Packard Enterprise, and as Research Engineer at WorkOnward.
    
    \item \textbf{Significant Performance:} \drs significant performance is demonstrated through \ul{62 granted United States patents} in AI, storage systems, and computer science; development of generative AI systems as Research Engineer at WorkOnward; and peer-reviewed publications in AI/ML venues.
    
    \item \textbf{Reference Endorsements:} Three distinguished IEEE members attested to \drs contributions: \textbf{[Reference 1]} (Principal Engineer, Insulet---Advanced Algorithms for Automated Drug Delivery), \textbf{[Reference 2]} (Intel Leadership; \textbf{IEEE Computer Society Secretary 2024, Board of Governors 2024--2026}), and \textbf{[Reference 3]} (AI \& Data Science Team Manager, Kelly).
    
    \item \textbf{Technical Contributions:} \drs 62 U.S. patents, peer-reviewed publications, and conference presentations demonstrate sustained contributions far beyond routine work.
\end{enumerate}

\paragraph{Expert Judgment and Review Process}

IEEE employs rigorous peer-review for all Senior Member applications through the Admissions and Advancement (A\&A) Review Panel . The review process requires: (1) candidate submission of a detailed application with professional history, achievements, and reference endorsements; (2) IEEE verification of references from Senior Members or Fellows; (3) evaluation by the A\&A Review Panel composed of Senior Members and Fellows; (4) determination of whether achievements constitute ``significant performance''; and (5) elevation upon successful review.

\vspace{0.5em}
\textbf{\drs Application Outcome:} \dr submitted his Senior Member application with comprehensive documentation of his 62 U.S. patents and professional achievements. Three IEEE members provided endorsements as detailed above. Following review by the A\&A Panel, \dr was \ul{successfully elevated to IEEE Senior Member} (Member \#99106509) , and official elevation notification was received . His Senior Member status has been certified through December 2026 with a certificate signed by [IEEE President], 2026 IEEE President .

\vspace{0.5em}


\paragraph{Petitioner's Qualifications Evaluated for Membership}

The IEEE Review Panel evaluated the following achievements demonstrating \drs outstanding ability:

\vspace{0.5em}
\textbf{a. Technical Innovation:} \dr holds \ul{62 granted U.S. patents} in artificial intelligence, enterprise storage, cloud infrastructure, and data management---an extraordinary portfolio that places him among the most prolific inventors in his field. The USPTO grants patents only for novel, non-obvious inventions that advance the state of the art.

\vspace{0.5em}
\textbf{b. IEEE Conference Speaker Roles:} \drs expertise has been recognized through invitations to present at major IEEE conferences. At the \textbf{IEEE New Era AI World Leaders Summit 2025}, he was selected for \textit{two} presentations: ``A Novel Bio-Causal Agent-to-Agent Protocol (BCA2P) Framework'' (with Dr. Eric Wasiolek) and ``Design Specification and AI-Driven Digital Twin Architecture for Storage Devices.'' At the \textbf{IEEE Enterprise GenAI Summit 2025}, he presented ``AAA-IDE: Autonomous Agentic AI for Data Engineering.''

\vspace{0.5em}
\textbf{c. IEEE Impact Creator Recognition:} \dr has been recognized as an \ul{IEEE Impact Creator}---a select group of IEEE members who inspire a global community to innovate for a better future . IEEE Impact Creators are featured on the IEEE Transmitter platform, IEEE's curated content hub that provides an engineer's perspective on technology news and innovations. This designation acknowledges individuals who have made significant contributions to advancing technology for the benefit of humanity, which is IEEE's core mission. \drs selection as an IEEE Impact Creator demonstrates that IEEE recognizes him as a thought leader whose insights on AI, computing, and technology are worthy of sharing with the broader engineering community.

\vspace{0.5em}
\textbf{d. Technical Leadership:} \dr serves as \textbf{Chief Technology Officer at WorkOnward}, where he leads the development of generative AI-based solutions. Prior to this role, he held senior positions at \textbf{Dell Technologies} and \textbf{Hewlett Packard Enterprise}, industry leaders in enterprise computing and storage technology.

\vspace{0.5em}
\textbf{e. Academic Excellence:} \dr earned his \textbf{M.S. in Computer Science} from California State University, Long Beach (CSULB) in 2022--2024, where he was recognized with the \ul{CSULB Outstanding Graduate Student Award} for Research, Scholarly, and Creative Activities (2024), and also in the Dean List (2024). He holds a \textbf{B.S. in Computer Science and Engineering} from SRM University, Chennai, India (2005--2009).

\vspace{0.5em}
\textbf{f. Recognition by Field:} \dr serves as a \textbf{reviewer for premier venues} (CORE A Star conference) including NeurIPS, KDD, ACL, IEEE CSR (2024, 2025), and IEEE Access journal. He also serves as a \textbf{Mentor for the Sunstone Innovation Challenge} (2025) at the Institute for Innovation \& Entrepreneurship.

\paragraph{Evidence of Genuine Selectivity}

IEEE Senior Member is \ul{NOT} a fee-based or credential-based membership:
\begin{enumerate}
    \item \textbf{Application Required:} Comprehensive application with documented achievements
    \item \textbf{Expert References Required:} Three IEEE members (at least two Senior Members/Fellows) must attest to achievements
    \item \textbf{Review Panel Evaluation:} Every application evaluated by A\&A Panel
    \item \textbf{Rejections Occur:} Applications not demonstrating ``significant performance'' are denied
    \item \textbf{National Laboratory Recognition:} Oak Ridge National Laboratory (DOE) publicly announces Senior Member achievements as ``highest level of IEEE membership'' 
    \item \textbf{Statistical Selectivity:} Only 8--10\% of 460,000+ members achieve Senior Member status
\end{enumerate}

\paragraph{Independent Verification}

USCIS may verify \drs IEEE Senior Member status through:
\begin{enumerate}
    \item \textbf{IEEE Senior Membership Certificate} (Member \#99106509) , 
    \item \textbf{IEEE Elevation Notification} 
    \item \textbf{IEEE Member Directory:} \dr is listed as Senior Member
    \item \textbf{IEEE Conference Programs:} Speaking roles documented in official programs
    \item \textbf{USPTO Patent Database:} 62 U.S. patents independently verifiable
    \item \textbf{Contact:} IEEE Headquarters, 3 Park Avenue, 17th Floor, New York, NY 10016; +1 800 678 4333; member-services@ieee.org
\end{enumerate}

\paragraph{Additional IEEE Recognition and Activities}

\drs recognition by IEEE extends beyond Senior Member status. He serves as a \textbf{Certified Reviewer for IEEE CSR 2024 \& 2025} (Cyber Security and Resilience Conference) and as a \textbf{Peer Reviewer for IEEE Access}, a multidisciplinary open access journal publishing research across all IEEE fields of interest. Additionally, \dr holds memberships in the \textbf{IEEE Computer Society} (Senior Member status), the \textbf{IEEE Signal Processing Society}, and \textbf{IEEE Young Professionals}. These reviewer and committee roles demonstrate that IEEE considers \dr qualified to evaluate the work of others in the field---itself evidence of his standing as a recognized expert.

\vspace{0.5em}


%% ===========================================================================
%% ISSIP AMBASSADOR
%% ===========================================================================

\subsubsection{ISSIP Ambassador: Association Requiring Outstanding Achievements}
\label{member:issip}

\paragraph{Association Overview and Prestige}

The International Society of Service Innovation Professionals (ISSIP) is a \textbf{global community} that ``promotes understanding and excellence in service innovation to benefit people, business and society'' . In its official appointment correspondence, ISSIP further states that it is ``proudly associated with a league of distinguished organizations'' including \textbf{IEEE}, the \textbf{Association for the Advancement of Artificial Intelligence (AAAI)}, the \textbf{California Center for Service Science}, the \textbf{Cambridge Service Alliance UK}, and the \textbf{Swiss Institute of Service Science}, among others .

\vspace{0.5em}
\textbf{Relevance to \drs Field:} Service innovation directly overlaps with \drs field of extraordinary ability---artificial intelligence and computer science---because modern service systems are increasingly driven by AI-enabled products, platforms, and enterprise computing. \drs work in AI/ML systems and sustained technical innovation (as reflected in his U.S. patent portfolio) aligns with ISSIP's focus on advancing service innovation for societal benefit.

\paragraph{Membership Level and Selectivity}

\textbf{Membership Tier:} ISSIP Ambassador (appointed role)

\vspace{0.5em}
ISSIP's Ambassador Program was established in 2014 , and ISSIP appoints Ambassadors who liaise with peer organizations, initiatives, and conferences on ISSIP's behalf , . ISSIP publicly reported that it welcomed \textbf{34 new Ambassadors} on its January 29, 2025 Progress Call, bringing the \textbf{total to 120 current ISSIP Ambassadors} , . Compared to the thousands of professionals in ISSIP's member organizations (including IBM, Cisco, and university partners), this small cohort represents less than 1\% of the community, highlighting the select nature of the role.

\paragraph{Outstanding Achievement Requirements and Selection Process}

ISSIP Ambassador status is \ul{not} a routine or fee-based affiliation. ISSIP expressly states that Ambassadors are selected because of their ``impressive achievements and contributions'' and that the selection process is ``thorough,'' requiring: (1) \textbf{nomination}; and (2) \textbf{review and approval by ISSIP's Executive Leadership} . This process demonstrates that ISSIP Ambassador appointments are granted based on documented accomplishments and evaluated judgment by recognized leaders within the society.

\vspace{0.5em}
ISSIP further confirms that Ambassadors are entrusted with meaningful responsibilities, including the authority to propose ISSIP-sponsored initiatives, events, or activities, and to serve as connectors so that ISSIP members can contribute across ``sister associations'' .

\paragraph{Petitioner's Appointment and Independent Verification}

\textbf{\drs Appointment Outcome:} ISSIP President \textbf{[ISSIP President]} issued a formal appointment letter dated \textbf{July 4, 2024} confirming \drs selection and appointment as an ISSIP Ambassador . ISSIP also issued an official \textbf{Certificate of Appreciation} awarding \dr the ISSIP Ambassador designation , and provided welcome correspondence confirming the appointment .

\vspace{0.5em}
\textbf{Public Confirmation:} ISSIP's official blog announcement listing the 34 new Ambassadors welcomed on the January 29, 2025 Progress Call includes \dr by name . The same Ambassador announcement was also covered through an independent press release and syndicated news coverage , .

\vspace{0.5em}


%% ===========================================================================
%% CONCLUSION
%% ===========================================================================

\dr has established, through extensive documentation and independently verifiable evidence, that he holds memberships in \textbf{three associations} that: (1) are in his field of extraordinary ability; (2) require outstanding achievements for membership; and (3) judge those achievements through recognized national and international experts.

\vspace{0.5em}
\textbf{BCS Fellowship (FBCS)} represents recognition by the United Kingdom's chartered IT professional body that \dr has made outstanding contributions across body of work, professional impact, and standing in the community. The Royal Charter status of BCS and the statistical selectivity (less than 5\% of 65,000 members, approximately 2,500--3,000 Fellows worldwide per the 2024 Annual Report ) confirm this is elite recognition based on demonstrated excellence. His subsequent selection as CSULB IIE Mentor and the success of his mentee team CuraVoice at PyTorch 2025 demonstrate continued real-world impact.

\vspace{0.5em}
\textbf{IEEE Senior Member status} represents the independent judgment of recognized experts that \dr has achieved ``significant performance'' and made outstanding contributions to the field. The rigorous selection process, statistical selectivity (8--10\% of members), and recognition by U.S. DOE national laboratories as ``the highest level of IEEE membership'' demonstrate this is \ul{not} an honorary, fee-based, or credential-based membership, but elite recognition based on demonstrated excellence.

\vspace{0.5em}
\textbf{ISSIP Ambassador} represents recognition by a global professional society advancing service innovation, conferred through a nomination-based process with Executive Leadership review/approval . ISSIP's public reporting of \textbf{120 current Ambassadors} further evidences the limited, selective nature of the Ambassador cohort .

\vspace{0.5em}
\drs extraordinary record of \ul{62 U.S. patents}, combined with speaking roles at major IEEE conferences, IEEE Impact Creator recognition, BCS Fellowship, ISSIP Ambassador appointment, and successful mentorship leading to nationally recognized startup outcomes, collectively demonstrate sustained national and international acclaim. \textbf{This criterion is satisfied under both the initial evidentiary review and strongly supports the final merits determination that \dr is among the small percentage who have risen to the very top of the field.}

\vspace{0.5em}

